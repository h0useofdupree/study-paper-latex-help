%!TEX root = ../Thesis.tex
\section{Von \gls{bootdriveusb} bis \gls{workstation}}
\subsection{Einleitung}

Die Installation von Arch Linux ist ein Prozess, der Präzision und ein
fundiertes Verständnis der zugrundeliegenden Technologien erfordert. Die
folgenden Ausführungen bieten eine praxisnahe Anleitung zur Installation und
Konfiguration von Arch Linux, ergänzt durch beispielhafte Befehle, die später in
Codeblöcke integriert werden können.

\subsection{Bootmedium erstellen}

Der erste Schritt in der Installation von Arch Linux besteht darin, ein
bootfähiges Medium (\gls{bootdriveusb}) zu erstellen. Dies erfolgt typischerweise durch das
Herunterladen des neuesten Arch Linux ISO-Images von der offiziellen Website und
dessen Übertragung auf einen USB-Stick. Ein häufig genutztes Tool hierfür ist dd.
Beispielhaft könnte der Befehl zum Erstellen des Bootmediums wie folgt lauten:

\begin{figure}[bht]
    \begin{lstlisting}[caption={Command für \gls{bash}}]
       dd bs=4M if=path/to/archlinux.iso of=/dev/sdx status=progress
        oflag=sync 
    \end{lstlisting}
\end{figure}
Hierbei steht /dev/sdx für den Pfad des USB-Laufwerks. Es ist entscheidend, den
korrekten Laufwerkspfad zu wählen, um Datenverluste zu vermeiden.

\subsection{Installation und Partitionierung}

Nach dem Booten des Systems vom USB-Stick beginnt der eigentliche
Installationsprozess. Dieser umfasst die Netzwerkkonfiguration, die
Festplattenpartitionierung, die Formatierung der Partitionen und die
Installation des Grundsystems.

Die Partitionierung kann mittels fdisk oder parted erfolgen. Ein Beispielbefehl
zur Partitionierung einer Festplatte (/dev/sda) wäre: siehe Listing \ref{lst:fdisk}
\newpage
\begin{figure}[bht]
    \begin{lstlisting}[caption={Command für \gls{bash}}, label=lst:fdisk]
        fdisk /dev/sda
    \end{lstlisting}
\end{figure}