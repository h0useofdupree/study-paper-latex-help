%!TEX root = ../Thesis.tex
\section{Einleitung}

In der sich stetig wandelnden Welt der Informationstechnologie, steigt die Bedeutung einer produktiven und funktionalen Workstation immer weiter. Insbesondere in professionellen und akademischen Sektoren ist ein leistungsfähiger Arbeitsplatz unerlässlich. Zwischen dem Nutzer und seiner Hardware steht nur noch das \Gls{OS}.

Vom Nischenprodukt für Enthusiasten bis hin zur ersten Wahl vieler Neuankömmlinge, hat sich Linux sowohl in der Softwareentwicklung als auch für wissenschaftliches Rechnen durchgesetzt. Durch seinen \gls{FOSS}-Charakter ermöglicht Linux dem Nutzer eine unbegrenzte vielfalt an Konfigurationsmöglichkeiten.

Die Vielfalt der verfügbaren Distributionen --- von Ubuntu und Fedora bis hin zu
spezialisierten Versionen wie CentOS --- bietet Nutzern eine breite Palette an
Optionen, um ein System nach ihren Bedürfnissen zu gestalten.

Der Zweck dieses Papiers besteht darin, eine detaillierte Anleitung zur
manuellen Konfiguration eines produktiven Linux-Arbeitsplatzes zu bieten. Es
behandelt Themen wie die Auswahl einer geeigneten Distribution, die Installation
und Konfiguration von Software, Sicherheitsaspekte und Optimierungstechniken, um
eine effiziente Arbeitsumgebung zu schaffen. In diesem Fall, wird das die Distribution \enquote{Arch Linux} als Konfigurationsbeispiel verwendet, da dieses es erlaubt, Linux fast von Grund auf zu konfigurieren. Vom \gls{TTY} bis hin zur \gls{GUI} mit einem produktiven \gls{tiling} \gls{windowmanager}

Dieses Papier richtet sich an eine breite Zielgruppe, von IT-Professionellen bis
hin zu Studierenden der Informatik, und bietet wertvolle Einblicke und
Anleitungen für jeden, der Interesse an der Einrichtung eines produktiven, auf
Linux basierenden Arbeitsplatzes hat.