%!TEX root = ../Thesis.tex

\section{Theoretischer Rahmen}
\subsection{Geschichte und Entwicklung von Linux}
Im Jahr 1991 neigte sich die Ära des Kalten Krieges dem Ende zu und leitete eine Zeit des Friedens und der Ruhe ein. Diese Periode war auch im Bereich der
Computertechnologie von großer Bedeutung. Die Leistungsfähigkeit der
Computerhardware überschritt alle Erwartungen, aber es gab immer noch eine Lücke
- die der Betriebssysteme.\footcite[1]{Hasan2004}

Während DOS von Bill Gates, das von einem Hacker aus Seattle für 50.000 Dollar
erworben wurde, aufgrund seiner cleveren Marketingstrategie die Welt der
persönlichen Computer dominierte, waren Apple Macintosh-Computer aufgrund ihrer
hohen Preise für die meisten Menschen unerreichbar. Gleichzeitig war das
Unix-Betriebssystem, eine andere wichtige Plattform, für kleine PC-Nutzer zu
teuer. Die Unix-Anbieter hatten den Quellcode, der einst in Universitäten
gelehrt wurde, zurückgehalten, was die Frustration der PC-Nutzer weltweit
erhöhte.\footcite[1]{Hasan2004}

In dieser Zeit erschien MINIX, ein von Andrew S\@. Tanenbaum, einem
niederländischen Professor, entwickeltes Betriebssystem. Tanenbaum schrieb MINIX,
um seinen Studenten die Funktionsweise eines echten Betriebssystems
näherzubringen. Obwohl MINIX selbst nicht herausragend war, war sein Quellcode
verfügbar, was es Programmieranfängern und Hackern ermöglichte, zum ersten Mal
in die Quellcodes eines Betriebssystems einzutauchen. Dies weckte das Interesse
von Informatikstudenten weltweit, unter ihnen auch Linus Torvalds.\footcite[1]{Hasan2004} 

\newpage

\subsection{Grundlagen von Betriebssystemen und Linux}
Linux, ein Kernstück der modernen Computertechnologie, steht im Zentrum
zahlreicher Innovationen und Entwicklungen im Bereich der Betriebssysteme.
\Glspl{OS} selbst sind die grundlegenden Softwarekomponenten eines
jeden Computers, die als Vermittler zwischen der Hardware und den
Anwendungsprogrammen fungieren. Sie verwalten die Hardware-Ressourcen eines
Computers und bieten Benutzern eine Schnittstelle für die Interaktion mit dem
System.

Die Besonderheit von Linux liegt in seinem Status als
Open-Source-Betriebssystem. Entstanden in den frühen 1990er Jahren durch die
Arbeit von Linus Torvalds, einem finnischen Studenten, hat sich Linux zu einer
der wichtigsten Plattformen in der IT-Welt entwickelt. Im Gegensatz zu
proprietären Betriebssystemen wie Windows von Microsoft oder macOS von Apple,
ist der Quellcode von Linux für jeden zugänglich und kann von jedermann
modifiziert und verteilt werden. Diese Offenheit hat eine große Gemeinschaft von
Entwicklern und Nutzern geschaffen, die ständig an der Verbesserung und
Erweiterung des Systems arbeiten.

Ein weiteres Kernmerkmal von Linux ist seine Vielseitigkeit. Linux kann auf
einer Vielzahl von Hardwareplattformen eingesetzt werden, von Embedded-Systemen
und Mobilgeräten bis hin zu Supercomputern. Diese Flexibilität macht es zu einer
attraktiven Wahl für viele verschiedene Anwendungen. Darüber hinaus sind
Linux-Distributionen (oder ``Distros'') wie Ubuntu, Fedora und Debian in
verschiedenen Konfigurationen erhältlich, die auf unterschiedliche
Nutzerbedürfnisse zugeschnitten sind.

Die Architektur von Linux basiert auf dem Unix-System, das in den 1960er und
1970er Jahren von AT\&T's Bell Labs entwickelt wurde. Wie Unix besteht Linux aus
einem Kernel, der die Kommunikation zwischen Hardware und Software steuert,
sowie einer Sammlung von Software-Werkzeugen, die es dem Benutzer ermöglichen,
mit dem System zu interagieren. Linux unterstützt eine Vielzahl von
Dateisystemen, Netzwerkprotokollen und bietet robuste Sicherheitsfunktionen.

Eines der Schlüsselelemente, die zur Beliebtheit von Linux beigetragen haben,
ist seine starke Sicherheitsarchitektur. Linux-Systeme gelten als äußerst sicher
und sind weniger anfällig für Viren und Malware als viele andere Betriebssysteme.
Dies liegt teilweise an der Art und Weise, wie Linux Benutzerrechte verwaltet
und wie die Gemeinschaft schnell auf Sicherheitslücken reagiert.

Zusammenfassend ist Linux ein mächtiges, flexibles und sicheres Betriebssystem,
das in einer Vielzahl von Umgebungen eingesetzt wird. Seine Entwicklung und
anhaltende Verbesserung durch eine aktive Open-Source-Gemeinschaft machen es zu
einer fortlaufenden Quelle der Innovation in der Welt der Betriebssysteme.