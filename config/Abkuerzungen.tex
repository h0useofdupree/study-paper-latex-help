%Acronyms
\newacronym{AES}{AES}{Advanced Encryption Standard}
\newacronym{AI}{AI}{Artificial Intelligence}
\newacronym{API}{API}{Application Programming Interface}
\newacronym{OS}{OS}{Operating System}
\newacronym{FOSS}{FOSS}{Free and Open Source Software}
\newacronym{WM}{WM}{Window Manager}
\newacronym{GUI}{GUI}{Graphical User Interface}
\newacronym{TTY}{TTY}{teletypewriter}
\newacronym{USB}{USB}{Universal Serial Bus}

%Glossary
\newglossaryentry{Glossar}
{
	name=Glossar,
	description={Ein Glossar ist eine alphabetisch geordnete Liste von Begriffen aus einem bestimmten Wissensgebiet mit den dazugehörigen Definitionen.}
}

\newglossaryentry{Kernel}
{
	name=Kernel,
	description={Ein Kernel, auch Betriebssystemkern, ist der zentrale Bestandteil eines Betriebssystems. In ihm ist die Prozess- und Datenorganisation festgelegt, auf der alle weiteren Softwarebestandteile des Betriebssystems aufbauen.}
}
\newglossaryentry{Embedded-System}
{ 
	name={Embedded-System},
	description={Ein
Embedded System ist ein spezieller Typ von Computersystem, der für eine
bestimmte Aufgabe oder Funktion entwickelt wurde. Es ist in der Regel in einem
größeren System eingebettet und wird oft in Geräten wie Haushaltsgeräten,
Automobilen, medizinischen Geräten und Industrieanlagen eingesetzt. Embedded
Systems sind darauf ausgelegt, spezifische Aufgaben zuverlässig und effizient
auszuführen und haben in der Regel begrenzte Ressourcen, wie Prozessorleistung
und Speicher. Sie spielen eine entscheidende Rolle in der modernen Technologie
und sind in vielen Aspekten des täglichen Lebens weit verbreitet.}
}

\newglossaryentry{windowmanager}{ name={Window Manager (Linux)},
description={Ein Window Manager ist eine Softwarekomponente in Linux- und
Unix-basierten Betriebssystemen, die für das Management von Fenstern auf dem
Bildschirm verantwortlich ist. Er ermöglicht das Erstellen, Verschieben,
Minimieren und Maximieren von Fenstern sowie die Verwaltung der
Fensterdekorationen. Window Manager spielen eine wichtige Rolle bei der
Benutzeroberfläche von Linux-Systemen und bieten verschiedene Ansätze zur
Fensterverwaltung, von einfachen Tiling-Managern bis hin zu komplexen
Desktop-Umgebungen wie GNOME und KDE.} }

\newglossaryentry{gui}{ name={GUI}, description={Die Abkürzung GUI steht für
"Graphical User Interface" oder auf Deutsch "Grafische Benutzeroberfläche". Es
handelt sich um eine visuelle Methode zur Interaktion mit einem Computer oder
Softwareanwendungen. Eine GUI stellt grafische Elemente wie Fenster, Symbole,
Schaltflächen und Menüs bereit, um Benutzern die Navigation und Interaktion mit
dem System zu erleichtern. GUIs sind in der Regel intuitiver und
benutzerfreundlicher als textbasierte Benutzeroberflächen (CLI) und werden in
vielen Betriebssystemen und Anwendungsprogrammen eingesetzt.} }

\newglossaryentry{tty}{ name={TTY}, description={TTY steht für "Teletypewriter"
oder "TeleTYpewriter" und bezieht sich auf eine Art von textbasiertem Terminal
in Unix-ähnlichen Betriebssystemen wie Linux. Eine TTY-Sitzung ermöglicht es
Benutzern, textbasierte Befehle und Anwendungen auszuführen, ohne eine grafische
Benutzeroberfläche zu verwenden. TTYs sind besonders nützlich für Aufgaben wie
die Fernsteuerung von Systemen über eine SSH-Verbindung oder die Fehlerbehebung
auf Ebene der Befehlszeile. In Linux sind TTYs numerisch nummeriert, zum
Beispiel TTY1, TTY2, usw., und können über Tastenkombinationen wie "Ctrl+Alt+F1"
aufgerufen werden.} }
\newglossaryentry{tiling}
{
	name={Tiling},
	description={
		Tiling im Kontext \gls{WM} bedeutet, dass die Fenster in einem Rasterartigen Layout angezeigt werden und meist den vollen Bildschirm nutzen.
	}
}
\newglossaryentry{bootloader}{ name={Bootloader}, description={Ein Bootloader
ist ein kleines Programm oder eine Softwarekomponente, die beim Start eines
Computers oder anderer elektronischer Geräte ausgeführt wird. Die Hauptaufgabe
eines Bootloaders besteht darin, das Betriebssystem des Geräts zu initialisieren
und zu laden. Dies geschieht, indem der Bootloader den Speicher nach dem
Betriebssystem durchsucht und es in den Hauptspeicher (RAM) lädt, um die
Ausführung zu starten. Der Bootloader ist in der Regel der erste Code, der nach
dem Einschalten des Geräts ausgeführt wird, und spielt eine entscheidende Rolle
im Bootvorgang. Es gibt verschiedene Bootloader für verschiedene Plattformen und
Betriebssysteme, darunter auch solche für Computer, Smartphones und eingebettete
Systeme.} }

\newglossaryentry{xorg}{ name={X.Org}, description={X.Org, auch als X Window
System oder X11 bezeichnet, ist ein Open-Source-Fenstersystem und Protokoll, das
in Unix-ähnlichen Betriebssystemen wie Linux und BSD verwendet wird. Es dient
dazu, grafische Benutzeroberflächen (GUIs) bereitzustellen und die Interaktion
zwischen Benutzern und Anwendungen über Fenster, Mauszeiger und Tastatureingaben
zu ermöglichen. X.Org bietet eine grundlegende Grundlage für die Anzeige und das
Management von Fenstern auf dem Bildschirm, während Desktop-Umgebungen und
Fenstermanager weitere Funktionen und Dienstprogramme hinzufügen, um eine
komplette grafische Arbeitsumgebung bereitzustellen. Obwohl X.Org in den letzten
Jahren von moderneren Display-Protokollen abgelöst wurde, bleibt es eine
wichtige Komponente in vielen Unix-Systemen und ist für die Kompatibilität mit
älterer Software von Bedeutung.} }

\newglossaryentry{bootdriveusb}{ name={Bootdrive}, description={Ein
Bootdrive ist ein --- meist \gls{USB} --- Massenspeichergerät, das verwendet wird, um ein
Computerbetriebssystem oder eine bootfähige Software zu starten. Es enthält ein
bootfähiges Betriebssystem oder eine bootfähige Anwendung, die in der Regel auf
einem USB-Flash-Laufwerk oder einer USB-Festplatte gespeichert ist. Ein
Bootdrive (USB) ermöglicht es einem Computer, von diesem externen USB-Gerät zu
starten, anstatt von der internen Festplatte oder SSD. Dies ist nützlich, wenn
Sie beispielsweise ein Betriebssystem neu installieren, Diagnosewerkzeuge
ausführen oder auf ein anderes Betriebssystem zugreifen möchten, ohne die
interne Festplatte zu ändern. Die Auswahl des Startgeräts erfolgt normalerweise
über das BIOS oder UEFI des Computers, wodurch der Computer vom Bootdrive (USB)
gestartet wird.} }

\newglossaryentry{workstation}{ name={Workstation}, description={Eine
Workstation ist ein leistungsstarker und spezialisierter Computer, der für
anspruchsvolle Aufgaben im Bereich der Wissenschaft, Grafikdesign,
Softwareentwicklung, 3D-Modellierung und anderen rechenintensiven Anwendungen
konzipiert ist. Workstations zeichnen sich durch ihre hohe Rechenleistung,
erweiterte Grafikfähigkeiten und die Fähigkeit zur Verarbeitung großer
Datenmengen aus. Sie sind in der Regel mit professionellen Grafikkarten,
schnellen Prozessoren und großem Arbeitsspeicher ausgestattet. Workstations
werden häufig von Fachleuten in Bereichen wie CAD (Computer-Aided Design),
Filmproduktion, wissenschaftlicher Forschung und Entwicklung, sowie in
Unternehmen eingesetzt, die anspruchsvolle rechenintensive Aufgaben bewältigen
müssen.} }

\newglossaryentry{bash}{ name={Bash}, description={Bash steht für "Bourne-Again
Shell" und ist eine weit verbreitete und leistungsstarke Unix-Shell, die auf
vielen Unix-ähnlichen Betriebssystemen, einschließlich Linux und macOS,
verwendet wird. Die Bash-Shell dient als Benutzerschnittstelle zur Interaktion
mit dem Betriebssystem über die Befehlszeile. Sie ermöglicht Benutzern das
Ausführen von Befehlen, das Erstellen von Skripten und die Automatisierung von
Aufgaben in der Textumgebung. Die Bash bietet eine Vielzahl von Funktionen,
darunter Befehlsverlauf, Tab-Vervollständigung, Umgebungsvariablen, Pipelining
und die Möglichkeit, Skripte zu erstellen, um komplexe Aufgaben zu
automatisieren. Aufgrund ihrer Flexibilität und Leistung ist die Bash eine der
am häufigsten verwendeten Unix-Shells und ein unverzichtbares Werkzeug für
Systemadministratoren, Entwickler und fortgeschrittene Benutzer.} }